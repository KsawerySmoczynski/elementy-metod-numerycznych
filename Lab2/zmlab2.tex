\documentclass[12pt,a4paper]{article}

\usepackage{amsmath,amssymb,amsthm,amsfonts} 
\theoremstyle{definition}
\newtheorem{thm}{Twierdzenie}
\usepackage[polish]{babel}
\usepackage[utf8]{inputenc}
\usepackage[T1]{fontenc}
\usepackage{enumerate}
\usepackage[left=2.5cm,right=2.5cm,top=2.5cm,bottom=2.5cm]{geometry}



\begin{document}

\section*{Interpolacja wielomianowa}


\subsection*{Zadania}
\begin{enumerate}
	
\item Napisz funkcję, która dla zadanego wielomianu $w$, wektora węzłów $(x_0,x_1,\ldots, x_n)$ oraz wektora odpowiadających im wartości $(y_0,y_1,\ldots,y_n)$ sprawdza, czy podany wielomian jest wielomianem Lagrange'a interpolującym te dane.


\item (* 4 pkt) Napisz funkcję, która dla wektora $n+1$ różnych punktów $(x_0,x_1,\ldots,x_n)$ i wartości pewnej funkcji $f$ w tych punktach zwraca wektor $(b_0,b_1,\ldots,b_n)$  współczynników wielomianu interpolacyjnego Lagrange'a funkcji $f$ w postaci Newtona opartego na węzłach $x_0,x_1,\ldots,x_n$.



\item (* 3 pkt) Napisz funkcję, która dla danych liczb rzeczywistych $a,b$ ($a<b$) i liczby naturalnej $n$ oblicza wartości $n+1$ węzłów Czebyszewa w przedziale $[a,b]$, czyli wartości:
$$x_j=\frac{b-a}{2}\cos(\frac{2j+1}{2n+2}\pi)+\frac{a+b}{2} \text{ dla } j=0,1,\ldots,n.$$

\item (* 2 pkt) Rozważmy funkcję $f(x)=\frac{1}{1+x^2}$ w przedziale $I=[-5,5]$.
\begin{enumerate}
	\item Znajdź współczynniki $b_i$ wielomianu interpolacyjnego Lagrange'a tej funkcji opartego na $6$ równoodległych węzłach w przedziale $I$.
	\item Znajdź współczynniki $b_i$ wielomianu interpolacyjnego Lagrange'a tej funkcji opartego na $11$ równoodległych węzłach w przedziale $I$.
	\item Narysuj w jednym oknie wykresy funkcji $f$ i dwóch obliczonych w poprzednich podpunktach wielomianów interpolacyjnych tej funkcji w przedziale $I$.
\end{enumerate}

\item (* 2 pkt) Dla funkcji $f$ z poprzedniego zadania wyznacz współczynniki wielomianów interpolacyjnych Lagrange'a w postaci Newtona opartych na $6$ i $11$ węzłach Czebyszewa w przedziale $[-5,5]$. Następnie narysuj w jednym oknie wykresy tych wielomianów i wyjściowej funkcji w tym przedziale.



\end{enumerate}




\end{document}